\section{Related Work}
For many years, the detection of mathematical formulas has been recognized as a difficult task \cite{Chan2000}. There are several existing methods for detecting mathematical expressions in PDF documents using formatting information, for example, page layout, character labels, character locations, font sizes, etc.  However, there are many different tools for generating PDF documents, and there is a variants in their character quality. Lin et al. \cite{Lin2011} show that mathematical formulas can be a composition of some object types. For instance, the square root sign in a PDF generated by \LaTeX consists of the text object representing a radical sign and a graphical object for the horizontal line, which results in the fact that some symbols must be identified from multiple drawings elements \cite{Mali2020}. \\
Lin et al. \cite{Lin2011} categorized the methods of formula detection into three types, based on the features they used, which are: character-based, image-based and layout-based methods. The first type, character-based methods use OCR to identify characters, and those which are not recognized by the OCR engine are considered candidates for mathematical expressions. The second category uses image segmentation, which we will not mention in this paper because our method process PDF documents only. The last one, layout-based methods use features such as line height, line spacing, alignment, etc to detect formulas. A lot of published papers use a combination of character, layout, and context features
\cite{Mali2020}.

\subsection{Traditional Methods}
Chaudhuri and Garain investigated over $10000$ document pages and found out the frequencies of each mathematical character in formulas \cite{Chaudhuri1998AnAF}. These frequencies are used to develop a detector for embedded expressions, which scans each text line and decides if the line consists of one of the $25$ most frequent characters. After finding the leftmost word that contains a mathematical symbol, the detector enlarges the region around the word on the left and right with rules to find the formula region.


There is another method based on mathematical symbols location, and then growing formula regions around the symbols. This method used fuzzy logic, which was developed by Kacem et al. \cite{Kacem2001}.

In 2011, Lin et. al proposed a four-step detection process, which finds embedded math formulas by merging characters tagged as math characters \cite{Lin2011}. SVM classification was used for both character classification into math and non-math, and isolated math formulas detection.

\subsection{CRF and Deep Learning-Based Methods}
For digital PDF documents, in 2017, Iwatsuki et al. \cite{Iwatsuki2017} developed a manually annotated dataset and applied conditional random fields (CRF) for detecting the zone of math expressions using both layout features (such as font types) and linguistic features (such as n-grams) extracted from PDF documents.

In 2017 also, Gao et al. \cite{Gao2017} published a mathematical formula detection method that combined CNN and RNN, which performs both top-down layout analysis based on XY-cutting, and bottom-up layout analysis based on connected components to generate formula region candidates.

In 2020, Mali et al. \cite{Mali2020} presented ScanSSD - Scanning Single Shot Detector for Mathematical Formulas in PDF Document Images. This method used only visual features for detection, which located formulas at multi scales using sliding windows, after which candidate detections are pooled to obtain page-level results.

Recently, Zhong et al.\cite{1stprize} proposed a method for detecting mathematical expressions using Generalized Focal Loss, an anchor-free method, instead of an anchor-based method and they found some tricks that were effective in this task. Their method was ranked 1st place among 15 teams in the final of the 2021 ICDAR Competition on Mathematical Formula Detection. We found the dataset as well as ideas for our method here.

